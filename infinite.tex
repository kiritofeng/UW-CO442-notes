\documentclass[main.tex]{subfiles}
\begin{document}
\section{Random Infinite Graphs}\lecture{Tue Nov 28}\mnote{This will not be on the exam.}%
Let $G$ be a graph on vertex set $\bN$
obtained by flipping a fair coin for each edge.
\begin{fact*}
  The graph contains a triangle.
\end{fact*}
\begin{proof}
  The chance a triple of vertices is not an triangle is $\frac 7 8$,
  hence the chance $n$ disjoint triples have no triangle is
  $\left(\frac 7 8\right)^n$ and taking $n$ to infinity shows this probability
  goes to 0.
\end{proof}
\begin{definition*}[Extension property, extensible, extends]
  We say a (possible infinite) graph $G = (V, E)$ has the
  \vocab{extension property} if, for all pairs $(A, B)$ of disjoint finite
  subsets of $V$, there is a vertex $x\notin A\cup B$ such that
  $A\subseteq N_G(x)$, $B\cap N_G(x) = \varnothing$.

  We say $(A,B)$ is \vocab{extensible} by $x$, which \vocab{extends} $(A,B)$.
\end{definition*}
It turns out every random graph with a countably infinite vertex set almost
certainly has the extension property.
\begin{proposition*}
  If $G$ is a graph with countably infinite vertex set constructed by flipping
  a coin with probability $0 < p < 1$ for every edge,
  then $\bP[G\text{ has the extension property}] = 1$.
\end{proposition*}
\begin{proof}
  It suffices to show for all $\eps > 0$,
  $\bP[G\text{ has an inextensible pair }(A, B)] < \eps$.

  For each particular pair $(A, B)$ and each $\delta > 0$, given $x$,
  $\bP[x\text{ extends }(A,B)] = p^{|A|}(1-p)^{|B|}$.
  So given distinct $x_1,\ldots,x_t\notin A\cup B$,
  \[
    \bP[\text{none of }x_1,\ldots,x_t\text{ extends }(A,B)]
    = \left(1 - p^{|A|}(1-p)^{|B|}\right)^t\leq\delta
  \]
  for large enough $t$.

  Let $(A_1, B_1), (A_2,B_2), \ldots$ be a list of a pair of disjoint finite
  subsets of $V$.
  Given $\eps > 0$, we have $\bP[(A_i,B_i)\text{ is extensible}]\leq\frac{\eps}{2^i}$
  so
  \begin{align*}
    \bP[\text{some pair is inextensible}]
    &\leq\sum_{i\geq 1}\bP[(A_i,B_i)\text{ is inextensible}] \\
    &\leq\sum_{i\geq 1}\frac{\eps}{2^i} = \eps. \qedhere
  \end{align*}
\end{proof}
\begin{proposition}
  $G$ has the extension property iff for all disjoint finite $(A,B)$ there are
  infinitely many $x$ that extend $(A,B)$.
\end{proposition}
\begin{proof}
  \leavevmode\vspace{-0.25em}
  \begin{itemize}[leftmargin=*,labelindent=2.5em]
    \item[($\impliedby$)] Easy.

    \item[($\implies$)] Given $(A,B)$, let $x_1$ extend $(A,B)$, let
      $x_2$ extend $(A\cup\{x_1\}, B)$, $x_3$ extend $(A\cup\{x_1,x_2\}, B)$, etc.
      Now all of $x_1,x_2,x_3,\ldots$ extend $(A,B)$. \qedhere
  \end{itemize}
\end{proof}
\begin{proposition}
  If $G$ has the extension property, so does $(G - X)\setminus F$ for all
  finite $X\subseteq V$ and $F\subseteq E$.
\end{proposition}
\begin{proof}
  Use the previous proposition.
\end{proof}
\begin{proposition}
  For any finite or countable graph $H$, if $G$ has the extension property,
  then $H$ is isomorphic to an induced subgraph of $G$.
\end{proposition}
\begin{proof}
  We may assume $H$ is countable with vertex set $\bN$.
  It suffices to show that there is a nested sequence of embeddings
  $\varphi_k:\{0,\ldots,k-1\}\to V(G)$ such that each $\varphi_k$ embeds
  $H[\{0,\ldots,k-1\}]$ as an induced subgraph of $G$.

  Given such a $\varphi_k$ for some $k$, let
  $A = \{0,\ldots,k-1\}\cap N_H(k)$ and $B = \{0,\ldots,k-1\}\cap N_H(k)$.
  Let $x$ be a vertex of $G$ that extends $(\varphi_k(A), \varphi_k(B))$.
  Define $\varphi_{k+1}$ by setting $\varphi_{k+1}(i) = \varphi_k(i)$
  for $i\leq k-1$ and $\varphi_{k+1}(k) = x$.
\end{proof}
\begin{proposition}\mnote{It's not true that \textit{any} infinite cell will
  have the extension property.}%
  If $V_1,\ldots,V_t$ is a partition of $V$ for some graph $G = (V, E)$ with
  the extension property, then some induced subgraph $G[V_i]$ has the extension
  property.
\end{proposition}
\begin{proof}
  If not, then for all $i$ there is an inextensible pair $(A_i,B_i)$ in $G[V_i]$.
  Now $\left(\bigcup A_i, \bigcup B_i\right)$ is extensible by some $x$,
  we have $x\in V_j$ for some $j$.
  Now $x$ extends $(A_j,B_j)$ in $G[V_j]$.
\end{proof}
\begin{proposition}[Erd\H{o}s, Renyi, Ackermann, Rado]
  Any two countable graphs with the extension property are isomorphic.
\end{proposition}
The unique countably infinite graph with the extension property is called the
\vocab{Rado graph}.
\begin{proof}
  Let $G$ and $H$ have the extension property; we may assume that $V(G) = \bN = V(H)$.
  We show that, for any isomorphism $\varphi_0$ between $G$ and $H$ respectively,
  there is a nested sequence $\varphi_0, \varphi_1, \ldots$ of isomorphisms
  between finite induced subgraphs of $G$ and $H$ such that for all $k$, the set
  $\{0,\ldots,k-1\}\subseteq\opname{dom}(\varphi_k)\cap\opname{range}(\varphi_k)$.

  We can extend $\varphi_k$ to $\varphi_{k+1}$ by finding values for
  $\varphi_{k+1}(k)$ and $\varphi_{k+1}\inv(k)$ by using the extension property
  in $H$ and $G$, respectively.

  Once we have such a sequence, $\varphi:\bN\to\bN$ defined by $\varphi(k) = \varphi_{k+1}(k)$
  is an isomorphism from $G$ to $H$.
\end{proof}
\begin{summary*}
  So the Rado graph $G$ satisfies:
  \begin{itemize}
    \item $G$ is isomorphic to $R$ iff $G$ has the extension property (for all
      countable $G$);
    \item every countable graph is an induced subgraph of $R$;
    \item for every partition $V_1,\ldots,V_k$, there exists $i$ such that
      $R[V_i]\cong R$; and
    \item $R$ is ``the unique infinite random graph''.
  \end{itemize}
\end{summary*}
It can be shown the only graphs that are have a part isomorphic to itself
are the empty graph, infinite clique, and the Rado graph.
\begin{remark*}
  Note that doing this construction with probability $p$ gives expected edge
  density $p$.
  The fact that there is a ``unique'' infinite random graph is not a
  contradiction because you can think of the isomorphism as permutations of
  an infinite set, and permuting an infinite set is a ridiculously strong
  property.
\end{remark*}
The Rado graph can be built deterministically: let $V = \bN$, and let
$n\sim m$ iff $n\in m$ when viewed as a subset of $\bN$ (i.e. an infinite
binary string) or vice versa.
This has the extension property because you can construct the binary string
corresponding to $A\subseteq\bN$ and then encoding this as an integer.

\begin{definition*}[H-set]
  Call $S$ an \vocab{H-set} if either $s\in\bN$ or $s = \{t_1,\ldots,t_k\}$
  for some finite collection of H-sets $t_1,\ldots,t_k$.
\end{definition*}
As an example, $3$, $\{2,3\}$, and $\{2, 4, \{3, 2\}\}$ are H-sets.
Then the Rado graph can be constructed by letting $H$ be the set of H-sets
and $x\sim y$ if $x\in y$ or $y\in x$.

An alternate construction comes from number theory: let $V$ be the set of primes
congruent to $1\pmod 4$, and let $p\sim q$ iff $p$ is a quadratic residue mod $q$
(i.e. $x^2\cong p\pmod q$ has a solution $x$).
\begin{theorem}[Gauss]
  $p$ is a quadratic residue mod $q$ iff $q$ is a quadratic residue mod $p$
  if $p,q\equiv 1\pmod 4$.
\end{theorem}
Thus the construction is symmetric.
To show this has the extension property,
let $A = \{p_1,\ldots,p_s\}, B = \{q_1,\ldots,q_t\}$.
Let $x_1,\ldots,x_s$ be quadratic residues mod $p_i$ and
$y_1,\ldots,y_s$ be quadratic non-residues mod $q_i$.
By the Chinese remainder theorem there exists
$x\pmod{4p_1\cdots p_sq_1\cdots q_t}$ that is $1\pmod 4$, $x_i\pmod{q_i}$,
$y_i\pmod q_i$.
By Dirichlet's theorem there is a prime equivalent to $x$ mod $4\prod p_i\prod q_i$.

\end{document}

