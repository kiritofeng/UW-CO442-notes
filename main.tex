\documentclass{scrartcl}
\usepackage[utf8]{inputenc}
\usepackage[english]{babel}
\usepackage[fancy,index,noauthor,nothm,physics,sexy]{kirito}
\usepackage[mdthm,secthm]{kiritothm}
\usepackage{marginnote,nicefrac,nicematrix,subfiles,tikz}
\usepackage[reftex]{theoremref}
\usetikzlibrary{calc,shapes.geometric}
\author{Roger Fu}
\date{2023 Fall Term}
\title{CO442 Notes}
\newcommand*{\mnote}[2][0pt]{\marginnote{\footnotesize #2}[#1]}
\newcommand*{\rmnote}[2][0pt]{\reversemarginpar{}\marginnote{\footnotesize #2}[#1]\normalmarginpar{}}
\newcommand*{\lecture}[1]{\rmnote{\ttfamily #1}}
\newcommand{\ul}[1]{\underline{#1}}

\begin{document}
\maketitle
\begin{abstract}
  This is the University of Waterloo's CO442 course, taught by P. Nelson.
  The formal name for this course is ``Graph Theory''.
  All errors are my own.
  The most recent version can be found at \url{https://github.com/kiritofeng/UW-CO442-notes}.
\end{abstract}

\tableofcontents
\clearpage
\section{Introduction}\lecture{Thu Sep 07}
\paragraph{Extremal theory} is when you are minimizing / maximizing a numerical
quantity of a graph (e.g. chromatic number, degree, number of edges)
subject to various constraints.

\begin{question*}
  Give a fixed graph $H$, how many edges can a graph $G$ with $n$ vertices have
  if $G$ has no $H$-subgraph?
\end{question*}
\begin{itemize}
  \item When $H$ is a triangle, this is Mantel's theorem.
  \item When $H = K_n$, this is Tur\'an's theorem.
  \item For ``all'' $H$, this is given by the Erd\H{o}s-Stone theorem
    (but this is an asymptotic result).
    We will prove this using the Szemer\'edi Regularity lemma.
\end{itemize}

\begin{question*}
  Give a fixed graph $H$, how many edges can a graph $G$ with $n$ vertices have
  if $G$ has no $H$-minor?
\end{question*}

\paragraph{Ramsey Theory} asks how many vertices are required so that every graph
contains either $t$ pairwise adjacent vertices or $t$ pairwise non-adjacent
vertices?
It's not immediately obvious that this number exists (i.e. what if there exists
very large graphs that have neither?)
This can also be extended to $k$-uniform hypergraphs.

\paragraph{Graph colouring} asks what the minimum number of distinct colours
are required to colour the vertex set of a graph such that no two adjacent
vertices are monochromatic.
In this class we will cover:
\begin{itemize}
  \item Brook's theorem: this gives a bound for $\chi(G)$ in terms of $\Delta(G)$;
  \item graphs on surfaces (potentially).
\end{itemize}

A related problem is edge colourings (i.e. colouring the edges such that no
two incident edges are monochromatic).
This called the \vocab{chromatic index} or \vocab{edge chromatic number}
(denoted $\chi'(G)$), and Vizing's theorem gives a bound on this quantity.

Another related problem is when every vertex has a list of permissible colours,
and you have to find a proper vertex colouring.
This gives rise to the \vocab{list chromatic number} $\chi^\ell(G)$
(and this is at most 5 for planar graphs, proved by Thomassen),
which is the minimum size of over all lists to get a proper vertex colouring.

A relaxation of proper colours is \textbf{defective colouring}, which is a
vertex colouring where all monochromatic edges form a matching.

\paragraph{Flows} ask about the assignment of directions and values such that
every vertex has the same amount of flow coming in and going out.
In particular, we will consider group-valued flows.
We can also count the number of flows for a given group.

A \vocab{nowhere zero flow} is a flow where none of the edges have 0 flow;
these are related to colourings.

\subsection{Definitions and Notation}
\begin{definition*}[Graph]
  A \vocab{graph} $G = (V, E)$ is a pair where $V$ is finite and
  $E\subseteq\binom{V}{2}$.
\end{definition*}

We using the following notation.
\begin{notation*}
  \listhack
  \begin{itemize}
    \item For an edge $e = \{x,y\}$ we will denote this $e = xy = yx$.
    \item We will write $x\sim y$ (or $x\sim_G y$ if we want to
      emphasize the graph) if $xy\in E(G)$.
    \item We use $d(x)$ (or $d_G(x)$) for the degree of $x$.
    \item We use $N(x)$ or $N_G(x)$ to denote the set of neighbours of $x$;
      in particular $x\notin N(x)$ (as the graphs in this course are loopless).
    \item We say $X\subseteq V(G)$ is \vocab{independent} if $X$ contains no edges.
    \item We say $X\subseteq V(G)$ is a \vocab{clique} if every pair of vertices
      in $X$ are adjacent.
    \item The natural numbers $\bN$ include 0.
  \end{itemize}
\end{notation*}

\subfile{extremal}
\subfile{ramsey}
\subfile{colouring}
\subfile{flows}
\subfile{infinite}
\clearpage
\printindex
\end{document}
