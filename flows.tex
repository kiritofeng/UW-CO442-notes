\documentclass[main.tex]{subfiles}
\begin{document}
\section{Nowhere-Zero Flows} \lecture{Tue Nov 14}\mnote{Thanks Chris T. for taking notes for the first 20 minutes.}
\begin{definition*}[$\Gamma$-preflow]
  Given a graph $G$ and an abelian group $\Gamma$,
  a \vocab[gamma-preflow@\2]{$\Gamma$-preflow} on $G$ is a tuple
  $\varphi = (\overline{\varphi}, \opname{head}, \opname{tail})$,
  where $\opname{head}: E \to V$, $\opname{tail}: E \to V$,
  and $\overline{\varphi} : E \to \Gamma$ such that
  $\{\opname{head}(e), \opname{tail}(e) \} = \{u, v\}$ for each edge $e$ with
  ends $u$ and $v$.
\end{definition*}
Write $\varphi(e)$ for $\overline{\varphi}(e)$ and $\varphi(X)$ for
$\sum_{e \in X} \varphi(e)$ for each $X \subseteq E(G)$.
\begin{notation*}
For each $X \subseteq V(G)$, let $\delta^+(x) = \{e : \opname{head}(e) \not \in X, \opname{tail}(e) \in X\}$ and $\delta^-(x) = \{e : \opname{tail}(e) \not \in X, \opname{head}(e) \in X\}$. Let $\delta^+(v) := \delta^+(\{v\})$, $\delta^-(v) := \delta^-(\{v\})$. 
\end{notation*}
We can now define flows and nowhere zero flows.
\begin{definition*}[$\Gamma$-flow, nowhere zero flow]
  \listhack
  \begin{itemize}
    \item We say $\varphi$ is a \vocab[gamma-flow@\2]{$\Gamma$-flow} on $G$ if
      $\varphi(\delta^+(v)) = \varphi(\delta^-(v))$ for all $v \in V(G)$.

    \item A \vocab[nonwhere zero flow]{nowhere zero $\Gamma$-flow} is a
      $\Gamma$-flow such that $\varphi(e)\neq 0$ for all $e$.
  \end{itemize}
\end{definition*}
We want to know where $G$ has such a flow.
\begin{question*}
  Given $\Gamma$ and $G$, does $G$ have a nowhere zero $\Gamma$-flow?
\end{question*}
We have the following characterization.
\begin{lemma}
  If $\varphi$ is a $\Gamma$-flow, then $\varphi(\delta^+(X)) = \varphi(\delta-(X))$
  for all $X\subseteq V(G)$.
\end{lemma}
\begin{proof}
  \leavevmode\vspace{-0.25em}
  \begin{align*}
    \varphi(\delta^+(X))
    &= \varphi(\delta^+(X)) + \sum_{v \in X} (\varphi(\delta^-(v)) - \varphi(\delta^+(v))) \\
    &= \sum_{\substack{e\in E\\\opname{head}(e)\in X\\\opname{tail}(e)\in X}}
          (\varphi(e) - \varphi(e))
    + \sum_{\substack{e\in E\\\opname{head}(e)\in X\\\opname{tail}(e)\notin X}} \varphi(e)
    + \sum_{\substack{e\in E\\\opname{head}(e)\notin X\\\opname{tail}(e)\in X}}
          (\varphi(e) - \varphi(e))\\
    &= \varphi(\delta^-(X)) \qedhere
  \end{align*}
\end{proof}
\begin{corollary}
  If $e$ is a bridge (i.e. edge such that $G-e$ has more components than $G$)
  of $G$, then $\varphi(e) = 0$ for every $\Gamma$-flow $\varphi$.
\end{corollary}
\begin{proof}
  For a flow $\varphi$, let $X$ be a set such that $\delta^+(X)  = \{e\}$,
  $\delta^-(X) = \varnothing$.
  Then
  \[
    \varphi(e) = \varphi(\delta^+(X)) = \varphi(\delta(X)) = \varphi(\varnothing)= 0.
    \qedhere
  \]
\end{proof}
\begin{question*}
  Given $\varphi$ and a bridgeless graph $G$, does $G$ have a nowhere zero
  $\Gamma$-flow?
\end{question*}
We give the characterizations for $\bZ_2$ and $\bZ_3$.
\begin{proposition}
  $G$ has a nowhere zero $\bZ_2$-flow iff every vertex has even degree.
\end{proposition}
\begin{proof}
  Left as an exercise to the reader.
\end{proof}
\begin{lemma}[Reversing edges]
  If $\varphi$ is a (nowhere zero) $\Gamma$-flow, $e$ is an edge of $G$,
  and $\varphi'$ is the preflow with $\varphi'(e) = -\varphi(e)$,
  $\opname{tail}'(e) = \opname{head}(e)$, $\opname{head}'(e) = -\opname{tail}(e)$,
  and $\varphi'$ agreess with $\varphi$ elsewhere, then $\varphi'$ is a (nowhere zero)
  $\Gamma$-flow.
\end{lemma}
\begin{proof}
  Left as exercise to the reader.
\end{proof}
Two flows $\varphi, \varphi'$ are \textit{equivalent} if each can be obtained
from the other by reversing edges.
\begin{proposition*}
  Let $G$ be a 3-regular graph.
  Then $G$ has a nowhere zero $\bZ_3$-flow iff $G$ is bipartite.
\end{proposition*}
\begin{proof}
  If $G$ has bipartition $(A,B)$, let $\varphi$ be the flow where
  $\varphi(e) = 1$ for all $e$ and each edge $e$ is directed from $A$ to $B$.

  If $v\in A$, then $\varphi(\delta^+(v)) - \varphi(\delta^-(v)) = (1 + 1 + 1) - 0 = 0$.
  If $v\in B$, then $\varphi(\delta^+(v)) - \varphi(\delta^-(v)) = 0 - (1+1+1) = 0$.
  Thus $\varphi$ is a nowhere zero $\bZ_3$-flow.

  Conversely, let $\varphi'$ be a nowhere zero $\bZ_3$-flow.
  Since $\varphi'(e)  \pm 1$ for all $e$, there is a nowhere zero $\bZ_3$-flow
  $\varphi$ equivalent to $\varphi'$ such that $\varphi(e) = 1$ for all $e$.

  Now, with respect to $\varphi$, for each $v$,
  \[
    0 = \varphi(\delta^+(v)) - \varphi(\delta-(v)) = |\delta^+(v)| - |\delta(v)|\bmod 3
  \]
  so $|\delta^+(v)|\equiv|\delta-(v)|\pmod 3$ for all $v$.

  Since $|\delta^+(v)| + |\delta^-(v)| = \deg(v) = 3$, it follows that each
  vertex either has out-degree 3 and in-degree 0, or vice-versa.

  Let $A = \{v : |\delta^+(v)| = 3\}, B = \{v : |\delta^-(v)| = 3\}$.
  We have just argued $A\cap B = \varnothing$, $A\cup B = V(G)$.
  Clearly neither $A$ nor $B$ contains an edge, so $(A,B)$ is a bipartition.
\end{proof}
\begin{theorem}[Tutte]
  Let $k\geq 1$.
  Then $G$ has z nowhere zero $\bZ_k$-flow iff $G$ has a $\bZ$-flow with values
  in $\{1,\ldots,k-1\}$.
\end{theorem}
\begin{proof}
  Let $\psi:\bZ\to\bZ_k$ be the natural homomorphism, and $r:\bZ_k\to\bZ$ be
  the (less) natural map.

  Suppose first that $G$ has a $\bZ$-flow $\varphi$ with values in $\{1,\ldots,k-1\}$.
  Let $\varphi'$ be the $\bZ_k$-preflow on $G$ with the same orientation as $\varphi$,
  and such that $\varphi'(e) = \psi(\varphi(e))$ for each $e$.
  Then, since $\varphi(e)\in\{1,\ldots,k-1\}$, we have $\psi(\varphi(e))\neq 0$
  so $\varphi'$ is never zero.

  For each $v\in V(G)$,
  \begin{align*}
    \varphi'(\delta^+(v)) - \varphi'(\delta-(v))
    &= \sum_{e\in\delta^+(v)}\psi(\varphi(e)) - \sum_{e\in\delta^-(v)}\psi(\varphi(e)) \\
    &= \psi\left(\sum_{e\in\delta^+(v)}\varphi(e) - \sum_{e\in\delta^-(v)}\varphi(e)\right)
      \tag{because $\psi$ is a homomorphism} \\
    &= \psi(\varphi(\delta^+(e)) - \varphi(\delta^-(e))) = \psi(0) = 0
  \end{align*}
  so $\varphi'$ is a flow.
  \mnote{Less formally, ``interpret'' $\varphi$ as being $\bZ_k$-valued.}

  Let $\varphi'$ be the $\bZ$-preflow on $G$ with the same orientations as
  $\varphi$ such that $\varphi'(e) = r(\varphi(e))$ for each edge.
  For each vertex $v$, let
  \begin{align*}
    d_{\varphi'(v)}
    &= \varphi'(\delta&+(v)) - \varphi'(\delta^-(v)) \\
    &= \sum_{e\in\delta^+(v)}r(\varphi(e)) - \sum_{e\in\delta^(v)}r(\varphi(e))
  \end{align*}
  and this deficiency is congruent to $0\pmod k$ by definition of $r$.
\end{proof}

\end{document}

